\begin{questions}
\Question{0}{TemplateAufgabe(Ana)}
Gegeben sei der NEA $N = (\{A,B,C,D,E,F\},\{0,1\},\delta,\{A\},\{A,F\})$ mit Übergangsfunktion $\delta$:\\

\begin{center}

    \begin{tikzpicture}[shorten >=1pt,node distance=2cm,on grid,auto, initial text=,every initial by arrow/.style=rectangle, accepting/.style=rectangle]

        \node[state, initial, rectangle] (A) {A};

        \node[state] (B) [right=of A] {B};

        \node[state] (C) [right=of B] {C};

        \node[state] (D) [below=of B] {D};

        \node[state] (E) [below=of C]{E};

        \node[state, accepting] (F) [right=of E] {F};

        

        \path[->]   (A) edge [bend left]        node        {0} (C)

                        edge                    node [swap] {1} (D)

                    (B) edge                    node        {0} (C)

                        edge [bend left]        node        {1} (D)

                    (C) edge [bend left]        node        {0} (E)

                        edge                    node        {0} (F)

                        edge [loop above]       node        {1} ()

                    (D) edge [bend left]        node        {0} (B)

                        edge [bend right]       node [swap] {0} (F)

                        edge                    node        {1} (C)

                    (E) edge                    node        {0} (D)

                        edge [bend left]        node        {1} (C);

    \end{tikzpicture}

\end{center}

Verwenden Sie die Potenzmengenkontruktion, um aus $N$ einen DEA $M$ mit $L(M) = L(N)$ zu erzeugen. Sie dürfen dabei nicht erreichbare Zustände vernachlässigen.
\pagebreak
\end{questions}
